\chapter{Specifikacija programske potpore}
		
	\section{Funkcionalni zahtjevi}
			\noindent \textbf{Dionici:}
			\begin{packed_enum}
				\item Pružatelji zdravstvenih usluga (klinike)
				\item Pružatelji prijevoznih usluga (prijevoznici)
				\item Klijenti zdravstvenog turizma
				\item Korisnici (administratori)
				\begin{packed_enum}
					\item[a)] administratori smještaja
					\item[b)] administratori prijevoza
					\item[c)] korisnički administratori
				\end{packed_enum}
				\item Razvojni tim
				
			\end{packed_enum}
			
			\noindent \textbf{Aktori i njihovi funkcionalni zahtjevi:}
			
			
			\begin{packed_enum}
				\item  \underbar{Neregistrirani/neprijavljeni korisnik (inicijator) može:}
				\begin{packed_enum}
					\item pregledati informacije o dostupnim tretmanima i klinikama u odabranoj državi i gradu, te vidjeti raspoloživost tih usluga
					\item odabrati dostupan tretman i poslati zahtjev za primanje tretmana
					\item se prijaviti na postojeći korisnički račun upisivanjem korisničkog imena i lozinke
					\begin{packed_enum}
						
						\item  podfunkcionalnost 1 
						\item  podfunkcionalnost 2
				
					\end{packed_enum}
				\end{packed_enum}
			
				\item  \underbar{Administrator smještaja (inicijator) može:}
				\begin{packed_enum}
					\item unijeti, modificirati i brisati podatke o smještaju
					\item vidjeti postojeće smještaje, njihove podatke i raspoloživost (uz grafički prikaz)
					\item unijeti i brisati podatke o prijavljenim pružateljima medicinskih usluga
					\item vidjeti postojeće prijavljene pružatelje medicinskih usluga
					\item vidjeti postojeće korisnike
					\item registrirati nove korisnike te dodjeljivati uloge i brisati postojeće
				\end{packed_enum}
				
				\item  \underbar{Administrator prijevoznih usluga (inicijator) može:}
				\begin{packed_enum}
					\item unijeti i brisati podatke o prijevoznicima
					\item modificirati podatke raspoloživosti prijevoznika
					\item vidjeti postojeće podatke prijevoznika i njihove raspoloživosti
				\end{packed_enum}
				
				\item  \underbar{Korisnički administrator (inicijator) može:}
				\begin{packed_enum}
					\item unijeti i brisati podatke korisnika medicinskih usluga
					\item vidjeti postojeće korisnike medicinskih usluga i njihove podatke
					\item preuzeti i vidjeti detalje tretmana klinika u kontekstu korisnika medicinskih usluga
				\end{packed_enum}
				
					\item  \underbar{Baza podataka (sudionik):}
				\begin{packed_enum}
					\item pohranjuje sve podatke o korisnicima i njihovim ovlastima
					\item pohranjuje sve podatke o smještajima, raspoloživosti i smještajnim kapacitetima
					\item pohranjuje sve podatke o prijevoznicima
					\item pohranjuje sve podatke o korisnicima medicinskih usluga
					\item pohranjuje sve podatke o dogovorenim terminima boravka pacijenta i prijevoza tijekom boravka
				\end{packed_enum}
			\end{packed_enum}
			\eject 
			
			
				
			\subsection{Obrasci uporabe}
				
				\textbf{\textit{dio 1. revizije}}
				
				\subsubsection{Opis obrazaca uporabe}
					\textit{Funkcionalne zahtjeve razraditi u obliku obrazaca uporabe. Svaki obrazac je potrebno razraditi prema donjem predlošku. Ukoliko u nekom koraku može doći do odstupanja, potrebno je to odstupanje opisati i po mogućnosti ponuditi rješenje kojim bi se tijek obrasca vratio na osnovni tijek.}\\
					

					\noindent \underbar{\textbf{UC$<$broj obrasca$>$ -$<$ime obrasca$>$}}
					\begin{packed_item}
	
						\item \textbf{Glavni sudionik: }$<$sudionik$>$
						\item  \textbf{Cilj:} $<$cilj$>$
						\item  \textbf{Sudionici:} $<$sudionici$>$
						\item  \textbf{Preduvjet:} $<$preduvjet$>$
						\item  \textbf{Opis osnovnog tijeka:}
						
						\item[] \begin{packed_enum}
	
							\item $<$opis korak jedan$>$
							\item $<$opis korak dva$>$
							\item $<$opis korak tri$>$
							\item $<$opis korak četiri$>$
							\item $<$opis korak pet$>$
						\end{packed_enum}
						
						\item  \textbf{Opis mogućih odstupanja:}
						
						\item[] \begin{packed_item}
	
							\item[2.a] $<$opis mogućeg scenarija odstupanja u koraku 2$>$
							\item[] \begin{packed_enum}
								
								\item $<$opis rješenja mogućeg scenarija korak 1$>$
								\item $<$opis rješenja mogućeg scenarija korak 2$>$
								
							\end{packed_enum}
							\item[2.b] $<$opis mogućeg scenarija odstupanja u koraku 2$>$
							\item[3.a] $<$opis mogućeg scenarija odstupanja  u koraku 3$>$
							
						\end{packed_item}
					\end{packed_item}
				
					
				\subsubsection{Dijagrami obrazaca uporabe}
					
					\textit{Prikazati odnos aktora i obrazaca uporabe odgovarajućim UML dijagramom. Nije nužno nacrtati sve na jednom dijagramu. Modelirati po razinama apstrakcije i skupovima srodnih funkcionalnosti.}
				\eject		
				
			\subsection{Sekvencijski dijagrami}
				
				\textbf{\textit{dio 1. revizije}}\\
				
				\textit{Nacrtati sekvencijske dijagrame koji modeliraju najvažnije dijelove sustava (max. 4 dijagrama). Ukoliko postoji nedoumica oko odabira, razjasniti s asistentom. Uz svaki dijagram napisati detaljni opis dijagrama.}
				\eject
	
		\section{Ostali zahtjevi}
		
			\textbf{\textit{dio 1. revizije}}\\
		 
			 \textit{Nefunkcionalni zahtjevi i zahtjevi domene primjene dopunjuju funkcionalne zahtjeve. Oni opisuju \textbf{kako se sustav treba ponašati} i koja \textbf{ograničenja} treba poštivati (performanse, korisničko iskustvo, pouzdanost, standardi kvalitete, sigurnost...). Primjeri takvih zahtjeva u Vašem projektu mogu biti: podržani jezici korisničkog sučelja, vrijeme odziva, najveći mogući podržani broj korisnika, podržane web/mobilne platforme, razina zaštite (protokoli komunikacije, kriptiranje...)... Svaki takav zahtjev potrebno je navesti u jednoj ili dvije rečenice.}
			 
			 
			 
	