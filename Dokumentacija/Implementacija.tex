\chapter{Implementacija i korisničko sučelje}
		
		
		\section{Korištene tehnologije i alati}

			 U timu smo za vrijeme projekta komunicirali ponajviše preko aplikacija Whatts up\footnote{\url{https://www.whatsapp.com/}} te Discorda\footnote{\url{https://discord.com/}}, a sa asistentom i demonstratorom preko Microsoft Teams\footnote{\url{https://teams.microsoft.com/v2/}} i Microsoft Outlook\footnote{\url{https://outlook.office.com/mail/}}. Tekst u dokumentaciji smo uređivali pomoću latex-a\footnote{\url{https://www.latex-project.org/}} u editoru texStudio\footnote{\url{https://www.texstudio.org/}} te generirali PDF dokument iz latex dokumenta pomoću TexLive\footnote{\url{https://www.tug.org/texlive/}}. Dijagrame uporabe te sekvencijske dijagrame smo izradili pomoću alata Astah UML\footnote{\url{https://astah.net/products/astah-uml/}}.
			 
			 Nedostaje za ostale dijagrame!!!!
			 
			 Model baze podataka je napravljen pomoću alata ERDplus\footnote{\url{https://erdplus.com/}}. Kako bi smo zajedno mogli raditi na projektu u isto vrijeme te ujedno i pratiti razvojne verzije našeg projekta smo koristili Git\footnote{\url{https://git-scm.com/}} zajedno za udaljenim repozitorijom GitHub\footnote{\url{https://github.com/}}. Razvojna okruženja koja su korištena su Visual Studio Code\footnote{\url{https://code.visualstudio.com/}} te pgadmin\footnote{\url{https://www.pgadmin.org/}}. Frontend je napisan u programskom jezika javascript\footnote{\url{https://www.javascript.com/}} pomoću biblioteke react\footnote{\url{https://react.dev/}}, dok je za backend korišten PostgreSQL\footnote{\url{https://www.postgresql.org/}}. Unit testovi su odrađeni pomoću alata
			 
			 nedostaje
			 
			 , dok su integracijski testovi napravljeni pomoću alata Selenium\footnote{\url{https://www.postgresql.org/}} i programskog  jezika
			 
			 Java\footnote{\url{https://www.java.com/en/}}????
			 
			 Aplikacija je puštena u pogon na servisu render\footnote{\url{https://render.com/}}.
			 
			 Fali za latex i texStudio
			
			\eject 
		
	
		\section{Ispitivanje programskog rješenja}
			
			\subsection{Ispitivanje komponenti}
		
			\subsection{Ispitivanje sustava}
			
			\eject 
		
		
		\section{Dijagram razmještaja}
			
			\textbf{\textit{dio 2. revizije}}
			
			 \textit{Potrebno je umetnuti \textbf{specifikacijski} dijagram razmještaja i opisati ga. Moguće je umjesto specifikacijskog dijagrama razmještaja umetnuti dijagram razmještaja instanci, pod uvjetom da taj dijagram bolje opisuje neki važniji dio sustava.}
			
			\eject 
		
		\section{Upute za puštanje u pogon}
		
			Aplikacija je puštena u pogon na cloud servisu Render čime smo omogućili javni pristup aplikaciji.
			
			\textbf{\textit{Konfiguracija frontenda}}
			
			\textbf{\textit{Konfiguracija baze/backenda}}
			
			
			\eject 