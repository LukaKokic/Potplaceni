\chapter*{Dodatak: Prikaz aktivnosti grupe}
		\addcontentsline{toc}{chapter}{Dodatak: Prikaz aktivnosti grupe}
		
		\section*{Dnevnik sastajanja}
		
		\begin{packed_enum}
			\item  sastanak
			
			\item[] \begin{packed_item}
				\item Datum:18.10.2023.
				\item Prisustvovali: Karlo Baljak, Luka Kokić, Ian Marković, Mateo Martić, Mislav Matić, Bruno Milaković, Teo Musa
				\item Teme sastanka:
				\begin{packed_item}
					\item  Međusobno upoznavanje
					\item  Rasprava o projektnom zadatku
				\end{packed_item}
			\end{packed_item}
			
			\item  sastanak
			
			\item[] \begin{packed_item}
				\item Datum:25.10.2023.
				\item Prisustvovali: Karlo Baljak, Luka Kokić, Ian Marković, Mateo Martić, Mislav Matić, Bruno Milaković, Teo Musa
				\item Teme sastanka:
				\begin{packed_item}
					\item  Podjela zadataka na projektu
					\item  Rasprava o alatima koje ćemo koristiti
				\end{packed_item}
			\end{packed_item}
			
			\item  sastanak
			
			\item[] \begin{packed_item}
				\item Datum:10.11.2023.
				\item Prisustvovali: Karlo Baljak, Luka Kokić, Ian Marković, Mateo Martić, Mislav Matić, Bruno Milaković, Teo Musa
				\item Teme sastanka:
				\begin{packed_item}
					\item  Rasprava o finalizaciji finijih detalja vezano uz obrasce uporabe i dijagrame te njihova finalizacija
					\item  Raspodjela zadataka koji trebaju biti obavljeni do prve predaje
				\end{packed_item}
			\end{packed_item}
			
			%
			
		\end{packed_enum}
		
		\eject
		\section*{Tablica aktivnosti}

			\begin{longtblr}[
					label=none,
				]{
					vlines,hlines,
					width = \textwidth,
					colspec={X[7, l]X[1, c]X[1, c]X[1, c]X[1, c]X[1, c]X[1, c]X[1, c]}, 
					vline{1} = {1}{text=\clap{}},
					hline{1} = {1}{text=\clap{}},
					rowhead = 1,
				} 
			
				\SetCell[c=1]{c}{} & \SetCell[c=1]{c}{\rotatebox{90}{\textbf{Luka Kokić}}} & \SetCell[c=1]{c}{\rotatebox{90}{\textbf{Karlo Baljak}}} &	\SetCell[c=1]{c}{\rotatebox{90}{\textbf{Ian Marković}}} & \SetCell[c=1]{c}{\rotatebox{90}{\textbf{Mateo Martić}}} &	\SetCell[c=1]{c}{\rotatebox{90}{\textbf{Mislav Matić}}} & \SetCell[c=1]{c}{\rotatebox{90}{\textbf{Bruno Milaković}}} &	\SetCell[c=1]{c}{\rotatebox{90}{\textbf{Teo Musa}}} \\  
				Upravljanje projektom 		&  &  &  &  &  &  & \\ 
				Opis projektnog zadatka 	&  &  &  &  &  &  & \\ 
				
				Funkcionalni zahtjevi       &  &  &  &  &  &  &  \\ 
				Opis pojedinih obrazaca 	&  &  &  &  &  &  &  \\ 
				Dijagram obrazaca 			&  &  &  &  &5  &  &  \\ 
				Sekvencijski dijagrami 		&  &  &  &  &  &  &  \\ 
				Opis ostalih zahtjeva 		&  &  &  &  &  &  &  \\ 

				Arhitektura i dizajn sustava	 &  &  &  &  &  &  &  \\ 
				Baza podataka				&  &  &  &  &  &  &   \\ 
				Dijagram razreda 			&  &  &  &  &  &  &   \\ 
				Dijagram stanja				&  &  &  &  &  &  &  \\ 
				Dijagram aktivnosti 		&  &  &  &  &  &  &  \\ 
				Dijagram komponenti			&  &  &  &  &  &  &  \\ 
				Korištene tehnologije i alati 		&  &  &  &  &  &  &  \\ 
				Ispitivanje programskog rješenja 	&  &  &  &  &  &  &  \\ 
				Dijagram razmještaja			&  &  &  &  &  &  &  \\ 
				Upute za puštanje u pogon 		&  &  &  &  &  &  &  \\  
				Dnevnik sastajanja 			&  &  &  &  &  &  &  \\ 
				Zaključak i budući rad 		&  &  &  &  &  &  &  \\  
				Popis literature 			&  &  &  &  &1  &  &  \\  
				&  &  &  &  &  &  &  \\ \hline 
				Dodatne stavke kako ste podijelili izradu aplikacije 			&  &  &  &  &  &  &  \\ 
				npr. izrada početne stranice				&  &  &  &  &  &  &  \\  
				izrada baze podataka		 			&  &  &  &  &  &  & \\  
				spajanje s bazom podataka							&  &  &  &  &  &  &  \\ 
				back end							&  &  &  &  &  &  &  \\  
			\end{longtblr}
					
					
		\eject
		\section*{Dijagrami pregleda promjena}
		
		\textbf{\textit{dio 2. revizije}}\\
		
		\textit{Prenijeti dijagram pregleda promjena nad datotekama projekta. Potrebno je na kraju projekta generirane grafove s gitlaba prenijeti u ovo poglavlje dokumentacije. Dijagrami za vlastiti projekt se mogu preuzeti s gitlab.com stranice, u izborniku Repository, pritiskom na stavku Contributors.}
		
	